\documentclass{article}
% Choose a conveniently small page size
% PACKAGES
\usepackage[margin = 1in]{geometry}
\usepackage{amsfonts}
\usepackage{amsmath}
\usepackage{amssymb}
\usepackage{multicol}
\usepackage{graphicx}
\usepackage{float}
\usepackage{xcolor}
\usepackage{amsthm}
\usepackage{dsfont}
\usepackage{hyperref}

% MACROS
% Set Theory
\def\N{\mathbb{N}}
\def\R{\mathbb{R}}
\def\C{\mathbb{C}}
\def\Z{\mathbb{Z}}
%\def\^{\hat}
\def\-{\vec}
\def\d{\partial}
\def\!{\boldsymbol}
\def\X{\times}
%\def\-{\bar}
\def\bf{\textbf}
\def\t{\text}
\def\b{\mathbf}
\def\l{\left}
\def\r{\right}
\title{Weekly Report}
\author{Damien}
\begin{document}
\maketitle
% \newpage
\section{Summary}

I am trying to fundamentally understand the methods for numerically approximating the collision kernel. I went through section 7.1 of "Summer Course at PKU (July 2020) Introduction to Kinetic Theory – Lecture Notes" to copy down your derivations and then went back to fill in any gaps that I was confused on. I successfully did this for section 7.1.1. I am still working on section 7.1.2 which is the derivation for the Carleman representation: the algorithm I am supposed to implement. 

\section{Progress}
I would like to note that some of the following equations are taken directly from Jingwei Hu's lecture notes to expedite the computation process. I add in some of my own calculations to simplify from the general Boltzmann equation to the case od 2D Maxwell molecules that we are concerned with. Recall that the full collision operator is given by
\begin{equation} \label{eq:collision}
    Q(f,f)(\b v) = \int_{\mathbb{R}^d} \int_{S^{d-1}} B(|\b v - \b v_*|,\cos \chi)[f(\b v_*') f(\b v') - f(\b v_*) f(\b v)] \t d \sigma \t d \b v_*.
\end{equation}

In the simpler case that we are working with we can simplifiy this expression to

\[
    Q(f,f)(\b v) = Q^+(f,f)(\b v) - C \rho f(\b v).
\]

I am figuring out how to appropriately compute $Q^+(f,f)$.
\begin{gather*}
    Q^+(f,f) = \frac{1}{2 \pi} \int_{\mathbb{R}^2} \int_{S^1} f(\b v') f(\b v'_*) \t d \sigma \t d \b v_*f(\b v') = \\
    \frac{1}{2 \pi} \int_{\mathbb{R}^2} \int_{S^1} f \left(\frac{\b v + \b v_*}{2} + \frac{\|\b v - \b v_*\| }{2} \sigma \right) f \left(\frac{\b v + \b v_*}{2} - \frac{\|\b v - \b v_*\| }{2} \sigma \right)\t d \sigma \t d \b v_*
\end{gather*}
We approximate this integral through the fast Fourier spectral method based on Carleman representation.
\subsection{General Fourier Spectral Methods}

Before discussing the Carleman representation we must first talk about the general Fourier-Galerkin spectral  methods for solving the spatially homogeneous Boltzmann equation. We must first truncate the problem. We choose to do this by apprroximating the solution on a torus: $\mathcal{D}_L = [-L,L]^d$.

\begin{equation} \label{eq:1}
    \begin{cases}
        \partial_t f = Q^R(f,f), \quad t > 0, v\in \mathcal{D}_L\\
        f(0, \b v) = f^0(\b v)
    \end{cases}
\end{equation}

The truncated collision operator is given by 

\[
    Q^R(g,f)(\b v) = \int_{\mathcal{B}_R} \int_{S^{d-1}} B_{\sigma}(|\b q|, \sigma \cdot \hat{\b q}) [ g(\b v'_*)f(\b v') - g(\b v-\b q)f(\b v)] \t d \sigma \t d \b q
\]

where a change of variable $\b v_* \to \b q = \b v - \b v_*$ is applied to the $\sigma$-representation of the collision operator, and the new variable $\b q$ is truncated to a ball $\mathcal{B}_R$ with radius $R$ centered at the origin. We write $\b q = |\b q|\hat{\b q}$ with $|\b q|$ being the magnitude and $\hat{\b q}$
being the direction. Accordingly,

\[
    \b v' = \b v - \frac{\b q - |\b q| \sigma}{2}, \quad \b v'_* = \b v - \frac{\b q + |\b q| \sigma}{2}.   
\]
For the 2D Maxwell molecules we then have that 

\[
    Q^{R+}(f,f)(\b v) = \frac{1}{2 \pi} \int_{\mathbb{R}^2} \int_{S^1} f \left( \b v - \frac{\b q - |\b q| \sigma}{2} \right) f \left( \b v - \frac{\b q + |\b q| \sigma}{2} \right) \t d \sigma \t d \b q
\]

In practice, the values of L and R are often chosen by an anti-aliasing argument: assume that $\text{supp}(f_0(v)) \subset \mathcal{B}_S$, then one can take \footnote{\textcolor{red}{I have no idea why $ R = 2S$ and $L \geq \frac{3 + \sqrt{2}}{2} S$.}}

\[
    R = 2S, \quad L \geq \frac{3 + \sqrt{2}}{2} S.
\]

 Given an integer $N \geq 0$, we then seek a truncated Fourier series expansion of $f$ as

\[
    f(t,\b v) \approx f_N(t,\b v) = \sum_{\b k \in \left\{-\frac{N}{2}, \frac{N}{2}\right\}^d} f_{\b k}(t) e^{\frac{i \pi}{L} \b k \cdot \b v } \in \mathbb{P}_N
\]

where 

\[
    \mathbb{P}_N = \t{span} \left\{ e^{\frac{i \pi}{L} \b k \cdot \b v } : \b k \in \left\{-\frac{N}{2}, \frac{N}{2}\right\}^d \right\},
\]

equipped with inner product

\[
    \langle f,g \rangle = \frac{1}{(2L)^d} \int_{\mathcal{D}_L} f \bar{g} \: d \b v\]

Substituting $f_N$ into \ref{eq:1} and conducting the Galerkin projection ($\mathcal{P}_N$) onto the space $\mathbb{P}_N$ yields

\begin{equation} \label{eq:2}
    \begin{cases}
        \partial_t f_N = \mathcal{P}_N Q^R(f_N,f_N), \quad t > 0, v\in \mathcal{D}_L\\
        f_N(0,\b v) = \mathcal{P}_N f^0(\b v)
    \end{cases}.
\end{equation}

$\mathcal{P}_N$ is defined as

\[
    \mathcal{P}_N [g(\b v)] = \sum_{\b k \in \left\{-\frac{N}{2}, \frac{N}{2}\right\}^d} \hat{g}_k(t) e^{\frac{i \pi}{L} \b k \cdot \b v }, \quad \hat{g}_k = \langle g, e^{\frac{i \pi}{L} \b k \cdot \b v } \rangle.
\]

Writing out each Fourier mode of \ref{eq:2}, we obtain

\begin{equation}
    \begin{cases}
        \frac{\t d}{\t d t} f_{\b k} = Q^R_{\b k}, \quad \mathbf{k} \in \{-\frac{N}{2}, \frac{N}{2}\}^d\\
        f_{\b k}(0) = f^0_{\mathbf{k}}
    \end{cases}
\end{equation}

with 

\[
    Q^R_{\b k} \equiv \langle Q^R(f_N, f_N), e^{\frac{i \pi}{L} \b k \cdot \b v } \rangle, \quad f^0_{\b k} \equiv \langle f^0, e^{\frac{i \pi}{L} \b k \cdot \b v} \rangle.
\]

Using the orthogonality of the Fourier basis, we can derive that
\begin{equation} \label{eq:QR}
    Q^R_{\b k} = \sum_{\b l,\b m \in \{-\frac{N}{2}, \frac{N}{2}\}^d} G(\b l,\b m) f_{\b l} f_{\b m} \quad \text{subject to} \quad \b l + \b m = k
\end{equation}

where G is given by

\begin{equation} \label{eq:Glm}
    G(\b l, \b m) = \int_{\mathcal{B}_R} \int_{S^{1}} B_{\sigma}(|\b q|, \sigma \cdot \hat{\b q}) \left[ e^{-\frac{i \pi}{2L} \b (\b l + \b m) \cdot \b q + \frac{i \pi}{2L} \b (\b l - \b m) \cdot \b \sigma} - e^{-\frac{i \pi}{L} \b m \cdot \b q } \right] \t d \sigma \t d \b q
\end{equation}

Ok, let's step away from blatantly plagiarizing Jingwei's notes for a moment to actually add something. The above steps were mystifying me, so I decided to go through the calculations to verify the form of $Q^R_{\b k}$. Before that however, we need to verify the form of $Q^R(f_N,f_N)$.

\begin{gather*}
    Q^R(f_N,f_N) = \int_{\mathcal{B}_R} \int_{S^{d-1}} B_{\sigma}(|\b q|, \sigma \cdot \hat{\b q}) \Bigg[ \left(\sum_{\b m \in \{\frac{N}{2}, \frac{N}{2} \}^d} f_{\b m}(t) e^{\frac{i \pi}{L} \b m \cdot \b v'_*} \right) \left(\sum_{\b l \in \{\frac{N}{2}, \frac{N}{2} \}^d} f_{\b l}(t) e^{\frac{i \pi}{L} \b l \cdot \b v'} \right) - \\
    \left(\sum_{\b m \in \{\frac{N}{2}, \frac{N}{2} \}^d} f_{\b m}(t) e^{\frac{i \pi}{L} \b m \cdot (\b v - \b q)} \right) \left(\sum_{\b l \in \{\frac{N}{2}, \frac{N}{2} \}^d} f_{\b l}(t) e^{\frac{i \pi}{L} \b l \cdot \b v} \right) \Bigg] \t d \sigma \t d \b q = \\
    \int_{\mathcal{B}_R} \int_{S^{d-1}} B_{\sigma}(|\b q|, \sigma \cdot \hat{\b q}) \sum_{\b m \in \{\frac{N}{2}, \frac{N}{2} \}^d} \sum_{\b l \in \{\frac{N}{2}, \frac{N}{2} \}^d} f_{\b m}(t) f_{\b l}(t) \left( e^{\frac{i \pi}{L} \b m \cdot \b v'_* + \frac{i \pi}{L} \b l \cdot \b v'}  - e^{\frac{i \pi}{L} \b m \cdot (\b v - \b q) + \frac{i \pi}{L} \b l \cdot \b v} \right) \t d \sigma \t d \b q = \\
    \int_{\mathcal{B}_R} \int_{S^{d-1}} B_{\sigma}(|\b q|, \sigma \cdot \hat{\b q}) \sum_{\b m \in \{\frac{N}{2}, \frac{N}{2} \}^d} \sum_{\b l \in \{\frac{N}{2}, \frac{N}{2} \}^d} f_{\b m}(t) f_{\b l}(t) \left( e^{\frac{i \pi}{L} \b m \cdot (\b v - \frac{\b q + |\b q| \sigma}{2}) + \frac{i \pi}{L} \b l \cdot \b (\b v - \frac{\b q - |\b q| \sigma}{2})}  - e^{\frac{i \pi}{L} \b m \cdot (\b v - \b q) + \frac{i \pi}{L} \b l \cdot \b v} \right) \t d \sigma \t d \b q = \\ 
    \int_{\mathcal{B}_R} \int_{S^{d-1}} B_{\sigma}(|\b q|, \sigma \cdot \hat{\b q}) \sum_{\b m \in \{\frac{N}{2}, \frac{N}{2} \}^d} \sum_{\b l \in \{\frac{N}{2}, \frac{N}{2} \}^d} f_{\b m}(t) f_{\b l}(t) \left( e^{\frac{i \pi}{L} \left( (\b m + \b l) \cdot \b v - (\b m + \b l) \cdot \b q + (\b l - \b m) \cdot \frac{|\b q| \sigma}{2} ) \right)}  - e^{\frac{i \pi}{L} \left( (\b m + \b l) \cdot \b v - \b m \cdot \b q \right)} \right) \t d \sigma \t d \b q = \\ 
    \int_{\mathcal{B}_R} \int_{S^{d-1}} B_{\sigma}(|\b q|, \sigma \cdot \hat{\b q}) \sum_{\b m \in \{\frac{N}{2}, \frac{N}{2} \}^d} \sum_{\b l \in \{\frac{N}{2}, \frac{N}{2} \}^d} f_{\b m}(t) f_{\b l}(t) e^{\frac{i \pi}{L} (\b m + \b l) \cdot \b v} \left( e^{\frac{i \pi}{L} \left( - (\b m + \b l) \cdot \b q + (\b l - \b m) \cdot \frac{|\b q| \sigma}{2} ) \right)}  - e^{- \frac{i \pi}{L} \b m \cdot \b q} \right) \t d \sigma \t d \b q \\ 
\end{gather*}

After doing this computation the form of Equation \ref{eq:QR} becomes apparent. All terms where $\b l + \b m \neq \b k$ are orthogonal to $e^{\frac{i \pi}{L} \b k \cdot \b v}$ and will disappear. Now, that we understand how to derive $Q^R_{\b k}$ we can continue on to get $Q^{R+}_{\b k}$ for the algorithm relevant to us. We can see that 

\[
Q^{R+}_{\b k} = \sum_{\b l,\b m \in \{-\frac{N}{2}, \frac{N}{2}\}^d} G^+(\b l,\b m) f_{\b l} f_{\b m} \quad \text{subject to} \quad \b l + \b m = \b k
\]
where 
\[
    G^+(\b l,\b m) = \frac{1}{2 \pi} \int_{\mathcal{B}_R} \int_{S^{1}} \sum_{\b m \in \{\frac{N}{2}, \frac{N}{2} \}^d} \sum_{\b l \in \{\frac{N}{2}, \frac{N}{2} \}^d} f_{\b m}(t) f_{\b l}(t) e^{\frac{i \pi}{L} \left( - (\b m + \b l) \cdot \b q + (\b l - \b m) \cdot \frac{|\b q| \sigma}{2} ) \right)}  \t d \sigma \t d \b q
\]

\subsection*{Fast Fourier Spectral Method Based on Carleman Representation}

The memory requirement and computational complexity of the direct Fourier spectral method may become a bottleneck when $N$ is large. We restate the general form of the collision operator here because I don't want to scroll all the way back up to look at it.

\[
    Q(g,f)(\b v) = \int_{\mathbb{R}^d} \int_{S^{d-1}} B(|\b v - \b v_*|,\cos \chi)[g(\b v_*') f(\b v') - g(\b v_*) f(\b v)] \t d \sigma \t d \b v_*.
\]

We apply the following change of variables.

\[
    \b v_* = \b v + \b x + \b y, \quad \b v' = \b v + \b x, \quad \b v'_* = \b v + \b y
\]

Plugging these into Equation \ref{eq:collision} and throwing in a $\delta(\b x \cdot \b y)$ to make sure we only consider cases where $\b v, \b v', \b v_*$, and $\b v'_*$ lie on a sphere. \footnote{\textcolor{red}{I don't understand the derivation for the Carleman representation. In particular, I want to see how the $\delta(\b x, \b y)$ pops out in the derivation. I am also not certain of the difference between $B_c$ and $B_\sigma$.}}

\[
    Q(g,f)(\b v) = \int_{\mathbb{R}^d} \int_{\mathbb{R}^d} B_c(\b x, \b y) \delta( \b x \cdot \b y)[g(\b v + \b y) f(\b v + \b x) - g(\b v + \b x + \b y) f(\b v)] \t d \b x \t d \b y.
\]
This is the so-called Carleman form. Now we follow the the same steps as above by plugging in $f_N$. 
\begin{gather*}
    Q(f_N,f_N)(\b v) = \int_{\mathbb{R}^d} \int_{\mathbb{R}^d} B_c(\b x, \b y) \delta( \b x \cdot \b y)\Bigg[ \left( \sum_{\b m \in \left\{-\frac{N}{2}, \frac{N}{2}\right\}^d} f_{\b m}(t) e^{\frac{i \pi}{L} \b m \cdot (\b v + \b y) } \right) \left( \sum_{\b l \in \left\{-\frac{N}{2}, \frac{N}{2}\right\}^d} f_{\b l}(t) e^{\frac{i \pi}{L} \b l \cdot (\b v + \b x) } \right)\\
    - \left( \sum_{\b m \in \left\{-\frac{N}{2}, \frac{N}{2}\right\}^d} f_{\b m}(t) e^{\frac{i \pi}{L} \b m \cdot (\b v + \b x + \b y) } \right) \left( \sum_{\b l \in \left\{-\frac{N}{2}, \frac{N}{2}\right\}^d} f_{\b l}(t) e^{\frac{i \pi}{L}(\b l \cdot \b v ) } \right) \Bigg] \t d \b x \t d \b y = \\
    \int_{\mathbb{R}^d} \int_{\mathbb{R}^d} B_c(\b x, \b y) \delta( \b x \cdot \b y)\Bigg[ \left( \sum_{\b m \in \left\{-\frac{N}{2}, \frac{N}{2}\right\}^d} \sum_{\b l \in \left\{-\frac{N}{2}, \frac{N}{2}\right\}^d} f_{\b m}(t) f_{\b l}(t) e^{\frac{i \pi}{L} (\b m \cdot (\b v + \b y) + \b l \cdot (\b v + \b x)) } \right)\\
    - \left( \sum_{\b m \in \left\{-\frac{N}{2}, \frac{N}{2}\right\}^d} \sum_{\b l \in \left\{-\frac{N}{2}, \frac{N}{2}\right\}^d} f_{\b m}(t) f_{\b l}(t) e^{\frac{i \pi}{L} (\b m \cdot (\b v + \b x + \b y) + \b l \cdot \b v) } \right) \Bigg] \t d \b x \t d \b y = \\
    \int_{\mathbb{R}^d} \int_{\mathbb{R}^d} B_c(\b x, \b y) \delta( \b x \cdot \b y) \sum_{\b m \in \left\{-\frac{N}{2}, \frac{N}{2}\right\}^d} \sum_{\b l \in \left\{-\frac{N}{2}, \frac{N}{2}\right\}^d} f_{\b m}(t) f_{\b l}(t) e^{\frac{i \pi}{L} (\b m + \b l) \cdot \b v} \left( e^{\frac{i \pi}{L} (\b m \cdot \b y + \b l \cdot \b x) } - e^{\frac{i \pi}{L} \b m \cdot (\b x + \b y) }\right) \t d \b x \t d \b y = 
\end{gather*}

From here on we will take $B_c(\b x, \b y) = \frac{1}{2 \pi}$. Using the orthogonality of the Fourier basis, we can derive that

\begin{equation} \label{eq:QR_carl}
    Q^R_{\b k} = \sum_{\b l,\b m \in \{-\frac{N}{2}, \frac{N}{2}\}^d} G(\b l,\b m) f_{\b l} f_{\b m} \quad \text{subject to} \quad \b l + \b m = \b k
\end{equation}

where G is given by \footnote{\textcolor{red}{The domain of integration in the following equation switched from a $\mathbb{R}^d$ to a $\mathcal{B}_r$. I am not sure why that is. Are they simply cutting out all of the things outside of that domain and accepting the error or is this equality?}}

\begin{equation} \label{eq:Glm_carl}
    G(\b l, \b m) = \int_{\mathcal{B}_R} \int_{\mathcal{B}_R} B_c(\b x, \b y) \left[  e^{\frac{i \pi}{L} (\b m \cdot \b y + \b l \cdot \b x) } - e^{\frac{i \pi}{L} \b m \cdot (\b x + \b y)} \right] \t d \b x \t d \b y.
\end{equation}

The idea of the fast algorithm is to find a separated expansion of the weight $G(\b l, \b m)$ (in fact, we only need to do this for the gain term because the loss term is readily a convolution) as

\[
    G^+(\b l, \b m) \approx \sum_{t = 1}^T \alpha_t(\b l) \beta_t(\b m),
\]

where T is small, so that the weighted convolution Equation \ref{eq:QR_carl} can be rendered into a few pure convolutions

\[
    Q_{\b k}^R+ \approx \sum_{t=1}^T \sum_{\b l,\b m \in \{-\frac{N}{2}, \frac{N}{2}\}^d} (\alpha_t(\b l) f_{\b l})(\beta_t(\b m) f_{\b m})  \quad \text{subject to} \quad \b l + \b m = \b k
\]

\section{To Do}

I found a good reference (Mouhot et al. \cite{Mouhot_2006}) that I am using to fill in the gaps of section 7.1.2. This is still a work in progress. I hope to finish the derivation for next week and make some headway on the code implementation.
\bibliographystyle{plain}
\bibliography{refs}
\end{document}